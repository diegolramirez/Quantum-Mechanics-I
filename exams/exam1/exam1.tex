\documentclass[11pt,letterpaper,addpoints]{exam}
% Choose language
\usepackage[spanish]{babel}

% Questions score
\boxedpoints
\pointname{ pt}

% Encoding options
\usepackage[T1]{fontenc}
\usepackage[utf8]{inputenc}
\usepackage{lmodern}

% Math and special stuff packages
\usepackage{amsmath}
\usepackage{mathtools}
\usepackage{amsfonts}
\usepackage{amssymb}
\usepackage{bigints}
\usepackage{graphicx}

% for listing with letters and changing number sizes
\usepackage{enumitem}
%\begin{enumerate}[label=\large{\textbf{\arabic*.}}]
%\begin{enumerate}[label=\large{\textbf{\alph*.}}]

% Selecting margins
\usepackage[left=1in,right=1in,top=1in,bottom=1in]{geometry}

% linking
\usepackage{hyperref}
\hypersetup{%
  colorlinks=true,% hyperlinks will be coloured
  urlcolor=blue
}

\title{Quantum Mechanics I EX1}
\author{Marek Nowakowski}
\date{\today}

\begin{document}
\maketitle

\begin{questions}
\question[5]{Consider a classical oscillator with

\begin{equation}
\label{X}
x(t) = A\cos(\omega t),\ A>0
\end{equation}

\begin{parts}
\part{Calculate $P(t) = mv(t)$. then calculate $E = \dfrac{P^2}{2m} + \dfrac{1}{2}m\omega^2x^2$ in terms of $A$ and \textit{vice versa} $A$ in terms of $E$.}
\part{Determine the space regions which are forbidden in the classical oscillator.}
\end{parts}}

\question[15]{Use 1(a) and 1(b) to calculate the quantum mechanical probability of a particle experiencing the harmonic force and having the energy $E = \dfrac{1}{2}\hbar\omega$ to be in the classically forbidden region.}

\end{questions}

\textbf{Help}
Wave function (properly normalized):

\begin{equation}
\label{wavefunc}
\Psi_n = \left( \frac{1}{\pi\lambda^2} \right)^{1/4} \frac{1}{\sqrt{2^nn!}}H_n\left( \frac{x}{\lambda} \right)e^{-x^2/2\lambda^2}
\end{equation}

\begin{equation}
\label{lambda}
\lambda = \sqrt{\frac{\hbar}{m\omega}}
\end{equation}

Hamiltonian operator

\begin{equation}
\label{hamiltonian}
\hat{H}\Psi_n = E_n\Psi_n
\end{equation}

This integral might be helpful as well

\begin{equation}
\label{integral}
1 - \frac{2}{\sqrt{\pi}}\bigintssss_0^1e^{-\eta^2}\mathrm{d}\eta = 0.1578
\end{equation}

% Count points
%\droptotalpoints

\begin{center}
\gradetable[h][questions]
\end{center}

\end{document}