\documentclass[11pt,letterpaper]{article}%letterpaper
% Choose language
\usepackage[english]{babel}

% Encoding options
\usepackage[T1]{fontenc}
\usepackage[utf8]{inputenc}
\usepackage{lmodern}

% Math and special stuff packages
\usepackage{amsmath}
\usepackage{mathtools}
\usepackage{amsfonts}
\usepackage{amssymb}
\usepackage{bigints}
\usepackage{graphicx}
\usepackage{siunitx}
\usepackage{empheq}
% for listing with letters and changing number sizes
\usepackage{enumitem}
%Use the following as arguments
%\begin{enumerate}[label=\large{\textbf{\arabic*.}}]
%\begin{enumerate}[label=\large{\textbf{\alph*.}}]

% Selecting margins
\usepackage[left=1in,right=1in,top=1in,bottom=1in]{geometry}


%%% NEW COMMANDS
%For boxing subequations in empheq environment 
\newcommand*\widefbox[1]{\fbox{\hspace{2em}#1\hspace{2em}}}

%Absolute Value
\newcommand\abs[1]{\left|#1\right|}

%%Inner product
\newcommand\inn[2]{\langle#1,#2\rangle}



%%%% CHANGING LABELS STYLE - may be used inside the document
%Sections labeled with roman numerals
\renewcommand\thesection{\Roman{section}}
%Equations labeled with roman numerals
%\renewcommand\theequation{\roman{equation}}
%Change equation counter
%\setcounter{equation}{4}

% Griffiths options - change false for true
\iffalse
% For setting r as in Griffiths
%\usepackage{graphicx}
\def\rcurs{{\mbox{$\resizebox{.16in}{.08in}{\includegraphics{ScriptR}}$}}}
\def\brcurs{{\mbox{$\resizebox{.16in}{.08in}{\includegraphics{BoldR}}$}}}
\def\hrcurs{{\mbox{$\hat \rcurs$}}}
\def\hbrcurs{{\mbox{$\hat \brcurs$}}}
\fi
% Declare hat vectors
% ijk
\newcommand{\ihat}{\hat{\textbf{\i}}}
\newcommand{\jhat}{\hat{\textbf{\j}}}
\newcommand{\khat}{\hat{\textbf{k}}}
% xyzs
\newcommand{\xhat}{\hat{\textbf{x}}}
\newcommand{\yhat}{\hat{\textbf{y}}}
\newcommand{\zhat}{\hat{\textbf{z}}}
\newcommand{\shat}{\hat{\textbf{s}}}
% r theta phi
\newcommand{\rhat}{\hat{\textbf{r}}}
\newcommand{\thhat}{\hat{\boldsymbol{\theta}}}
\newcommand{\phihat}{\hat{\boldsymbol{\phi}}}


% Differentials
\usepackage{upgreek}
\newcommand{\dl}{d\mathrm{\textbf{l}}}
\newcommand{\da}{d\mathrm{\textbf{a}}}
\newcommand{\dv}{d\uptau}


% Document information
\title{\textbf{Quantum Mechanics I - HW2}}
\author{Diego Ramírez Milano (201214691)}
\date{\today}

\begin{document}
\maketitle

The linear space of finite dimension (vectors) on which matrices act as linear operators is an example of a mathematical space (Hilbert Space) o which some Quantum Mechanics can be done.

\begin{enumerate}[label=\Large{\textbf{\arabic*.}}]\setcounter{enumi}{4}

\item{
Let $x$ and $\varphi$ be n-dimensional vectors (complex). We define the inner product $\inn{x}{\varphi}$ as

\begin{equation}
\label{inn}
\inn{x}{\varphi} \equiv x^\dag \varphi = \sum_i x_i^*\varphi_i
\end{equation}

Where $x_i$ and $\varphi_i$  are components of the vectors. Prove the following properties which $\inn{x}{\varphi}$ shares with the inner product defined for functions.

\begin{subequations}
\begin{empheq}{align}
	\label{inn1}\tag{\textit{i}}
	\inn{x}{x} &\geq 0\\
	\label{inn2}\tag{\textit{ii}}
	\inn{x}{x} &= 0 \iff x = 0\\
	\label{inn3}\tag{\textit{iii}}
	\inn{x}{\varphi} &= \inn{\varphi}{x}^*\\
	\label{inn4}\tag{\textit{iv}}
	\inn{x}{a\varphi} &= a\inn{x}{\varphi}\\
	\label{inn5}\tag{\textit{v}}
	\inn{ax}{\varphi} &= a^*\inn{x}{\varphi}\\
	\label{inn6}\tag{\textit{vi}}
	\inn{ax}{\varphi} &= \inn{x}{a^*\varphi}\\
	\label{inn7}\tag{\textit{vii}}
	\inn{x_1 + x_2}{\varphi} &= \inn{x_1}{\varphi} + \inn{x_2}{\varphi}
\end{empheq}
\end{subequations}

\begin{enumerate}[label=Proof for \textit{\roman*.}]
\item{
\begin{equation}
\label{inn1p}\tag{\textit{i proof}}
\begin{split}
\inn{x}{x} = \sum_i x_i^*x_i = \sum_i \abs{x_i}^2\\
\text{Since }\abs{x} \geq 0 \to \sum_i \abs{x_i}^2 \geq 0\\
\text{It follows } \inn{x}{x} \geq 0
\end{split}
\end{equation}
}

\item{
\begin{equation}
\label{inn2p}\tag{\textit{ii proof}}
\begin{split}
\inn{x}{x} = \sum_i \abs{x_i}^2\text{ but } \abs{x} = 0 \iff x = 0 \\
\text{Then it follows }\inn{x}{x} = 0 \iff x = 0
\end{split}
\end{equation}
}

\item{
\begin{equation}
\label{inn3p}\tag{\textit{iii proof}}
\inn{x}{\varphi} = \sum_i x_i^*\varphi_i = \left(\sum_i x_i^{**}\varphi_i^*\right)^* = \left(\sum_i x_i\varphi_i^*\right)^* = \left(\sum_i \varphi_i^*x_i\right)^* = \inn{\varphi}{x}^*
\end{equation}
}

\item{
\begin{equation}
\label{inn4p}\tag{\textit{iv proof}}
\inn{x}{a\varphi} = \sum_i x_i^*a\varphi_i = a\sum_i x_i^*\varphi_i = a\inn{x}{\varphi}
\end{equation}
}

\item{
\begin{equation}
\label{inn5p}\tag{\textit{v proof}}
\inn{ax}{\varphi} = \sum_i (ax_i)^*\varphi_i = \sum_i a^*x_i^*\varphi_i = a^*\sum_i x_i^*\varphi_i = a^*\inn{x}{\varphi}
\end{equation}
}

\item{
\begin{equation}
\label{inn6p}\tag{\textit{vi proof}}
\inn{ax}{\varphi} = \sum_i (ax_i)^*\varphi_i = \sum_i a^*x_i^*\varphi_i = \sum_i x_i^*(a^*\varphi_i) = \inn{x}{a^*\varphi}
\end{equation}
}

\item{
\begin{equation}
\label{inn7p}\tag{\textit{vii proof}}
\begin{split}
\inn{x_1 + x_2}{\varphi} &= \sum_i(x_1 + x_2)^*\varphi = \sum_i(x_{1i}^* + x_{2i}^*)\varphi = \sum_i\left( x_{1i}^*\varphi_i + x_{2i}^*\varphi_i \right)\\
&= \sum_i x_{1i}^*\varphi + \sum_i x_{2i}^*\varphi = \inn{x_1}{\varphi} + \inn{x_2}{\varphi}
\end{split}
\end{equation}
}
\end{enumerate}
\setcounter{equation}{1}
}

\item{
Let us write from now on in two dimensional vector space. Let $M$ be

\begin{equation}
\label{matrixM}
M = 
\begin{pmatrix}
m_1 & a\\
a^* & m_2
\end{pmatrix}
\text{, with } m_{i=1,2} \in \mathbb{R}
\end{equation}

\begin{enumerate}[label=\textit{\roman*.}]
\item{
Prove that $M = M^\dag$ ($M$ is hermitian) and $M^\dag = (M^T)^*$
\begin{equation}
\label{proofM}
M^\dag =
\begin{pmatrix}
m_1 & a^*\\
a & m_2
\end{pmatrix}^* =
\begin{pmatrix}
m_1 & a\\
a^* & m_2
\end{pmatrix} = M
\end{equation}
}

\item{
Find the eigenvalues $\lambda_i$ of $Mx=\lambda x$, where $x$ is a 2-dimensional vector.

\renewcommand\theequation{4.\alph{equation}}
\setcounter{equation}{0}
For any given square $2\times2$ of the form
\begin{equation}
\label{matrixA}
M = 
\begin{pmatrix}
a & b\\
c & d
\end{pmatrix}
\end{equation}

We can define the eigenvalues and eigenvectors, respectively, as follows

\begin{equation}
\label{eigenval}
\lambda_i = \frac{a+d}{2} \pm \sqrt{\frac{(a+d)^2}{4} - \left( ad - bc \right)}
\end{equation}

\begin{equation}
\label{eigenvec}
x_i = 
\begin{pmatrix}
\lambda_i - d\\
c
\end{pmatrix}
\end{equation}

Now using \eqref{eigenval} we find the eigenvalues for $M$ to be
\renewcommand\theequation{\arabic{equation}}
\setcounter{equation}{4}

\begin{equation}
\label{eigenvalues}
\begin{split}
\lambda_1 =& \frac{m_1 + m_2}{2} + \sqrt{\frac{(m_1 + m_2)^2}{4} - \left( m_1m_2 - \abs{a}^2 \right)}\\
\lambda_2 =& \frac{m_1 + m_2}{2} - \sqrt{\frac{(m_1 + m_2)^2}{4} - \left( m_1m_2 - \abs{a}^2 \right)}
\end{split}
\end{equation}
}

\item{
Prove that $\lambda_i$ are real

\begin{equation}
\label{realeigenval1}
\begin{split}
\lambda\inn{x}{x} =& \inn{x}{\lambda x}\text{ from relation \eqref{inn4}}\\
=& \inn{x}{Mx}\text{ from definition of eigenvalues and eigenvectors}\\
=& \inn{M^*x}{x}\text{ from relation \eqref{inn6}}\\
=& \inn{Mx}{x}\text{ from \eqref{proofM}}\\
=& \inn{\lambda x}{x}\text{ from definition of eigenvalues and eigenvectors}\\
=& \lambda^*\inn{x}{x}\text{ from relation \eqref{inn5}}
\end{split}
\end{equation}

Finally from \eqref{realeigenval1} and taking into account \eqref{inn1p} we get

\begin{equation}
\label{realeigenval2}
\lambda\inn{x}{x} = \lambda^*\inn{x}{x} \to \lambda = \lambda^* \to \lambda \in \mathbb{R}
\end{equation}
}

\item{
Find the eigenvectors $x_i$ (each corresponding to $\lambda_i$) such that are normalized to 1.

Again using \eqref{eigenvec} we find the eigenvectors of $M$ to be

\begin{equation}
\label{eigenvectors}
\begin{split}
x_1 =& 
\begin{pmatrix}
\frac{m_1 + m_2}{2} + \sqrt{\frac{(m_1 + m_2)^2}{4} - \left( m_1m_2 - \abs{a}^2 \right)} - m_2\\
a^*
\end{pmatrix}\\
x_2 =&
\begin{pmatrix}
\frac{m_1 + m_2}{2} - \sqrt{\frac{(m_1 + m_2)^2}{4} - \left( m_1m_2 - \abs{a}^2 \right)} - m_2\\
a^*
\end{pmatrix}
\end{split}
\end{equation}

The above vectors not normalized, so now we need to find each vector's norm and then divide them by the norm

\begin{equation}
\label{normV}
ghb
\end{equation}

}

\item{
Prove that $x_i^*x_j\ (i\neq j)$ is zero, meaning the eigenvectors are orthonormal.
}

\end{enumerate}

}


\end{enumerate}



\end{document}

%%%%%%%%% EXAMPLES %%%%%%%%%

%%%Example of boxed subequation%%%
%\begin{subequations}
%\begin{empheq}[box=\widefbox]{align}
%	\label{cos1}
%	&\cos(a) = \cos(b)\cos(c) + \sin(b)\sin(c)\cos(A) \\
%	\label{cos2}
%	&\cos(b) = \cos(a)\cos(c) \\
%	\label{cos3}
%	&\cos(c) = \cos(a)\cos(b) + \sin(a)\sin(b)\cos(C)
%\end{empheq}
%\end{subequations}