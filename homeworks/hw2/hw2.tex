\documentclass[11pt,letterpaper]{article}%letterpaper
% Choose language
\usepackage[english]{babel}

% Encoding options
\usepackage[T1]{fontenc}
\usepackage[utf8]{inputenc}
\usepackage{lmodern}

% Math and special stuff packages
\usepackage{amsmath}
\usepackage{mathtools}
\usepackage{amsfonts}
\usepackage{amssymb}
\usepackage{bigints}
\usepackage{graphicx}
\usepackage{siunitx}
\usepackage{empheq}
% for listing with letters and changing number sizes
\usepackage{enumitem}
%Use the following as arguments
%\begin{enumerate}[label=\large{\textbf{\arabic*.}}]
%\begin{enumerate}[label=\large{\textbf{\alph*.}}]

% Selecting margins
\usepackage[left=1in,right=1in,top=1in,bottom=1in]{geometry}


%%% NEW COMMANDS
%For boxing subequations in empheq environment 
\newcommand*\widefbox[1]{\fbox{\hspace{2em}#1\hspace{2em}}}

%Absolute Value
\newcommand\abs[1]{\left|#1\right|}

%%Inner product
\newcommand\inn[2]{\langle#1,#2\rangle}



%%%% CHANGING LABELS STYLE - may be used inside the document
%Sections labeled with roman numerals
\renewcommand\thesection{\Roman{section}}
%Equations labeled with roman numerals
%\renewcommand\theequation{\roman{equation}}
%Change equation counter
%\setcounter{equation}{4}

% Griffiths options - change false for true
\iffalse
% For setting r as in Griffiths
%\usepackage{graphicx}
\def\rcurs{{\mbox{$\resizebox{.16in}{.08in}{\includegraphics{ScriptR}}$}}}
\def\brcurs{{\mbox{$\resizebox{.16in}{.08in}{\includegraphics{BoldR}}$}}}
\def\hrcurs{{\mbox{$\hat \rcurs$}}}
\def\hbrcurs{{\mbox{$\hat \brcurs$}}}
\fi
% Declare hat vectors
% ijk
\newcommand{\ihat}{\hat{\textbf{\i}}}
\newcommand{\jhat}{\hat{\textbf{\j}}}
\newcommand{\khat}{\hat{\textbf{k}}}
% xyzs
\newcommand{\xhat}{\hat{\textbf{x}}}
\newcommand{\yhat}{\hat{\textbf{y}}}
\newcommand{\zhat}{\hat{\textbf{z}}}
\newcommand{\shat}{\hat{\textbf{s}}}
% r theta phi
\newcommand{\rhat}{\hat{\textbf{r}}}
\newcommand{\thhat}{\hat{\boldsymbol{\theta}}}
\newcommand{\phihat}{\hat{\boldsymbol{\phi}}}


% Differentials
\usepackage{upgreek}
\newcommand{\dl}{d\mathrm{\textbf{l}}}
\newcommand{\da}{d\mathrm{\textbf{a}}}
\newcommand{\dv}{d\uptau}
\newcommand{\dx}{\mathrm{d}x}
\newcommand{\dy}{\mathrm{d}y}
\newcommand{\dz}{\mathrm{d}z}


% Document information
\title{\textbf{Quantum Mechanics I - HW2}}
\author{Marek Nowakowski}
\date{\today}

\begin{document}
\maketitle

The linear space of finite dimension (vectors) on which matrices act as linear operators is an example of a mathematical space (Hilbert Space) on which some Quantum Mechanics can be done.

\begin{enumerate}[label=\Large{\textbf{\arabic*.}}]\setcounter{enumi}{4}

\item{
Let $x$ and $\varphi$ be n-dimensional vectors (complex). We define the inner product $\inn{x}{\varphi}$ as

\begin{equation}
\label{inn}
\inn{x}{\varphi} \equiv x^\dag \varphi = \sum_i x_i^*\varphi_i
\end{equation}

Where $x_i$ and $\varphi_i$  are components of the vectors. Prove the following properties which $\inn{x}{\varphi}$ shares with the inner product defined for functions $(f\circ g) = \bigintssss_{-\infty}^{\infty}\dx f^*g$.

\begin{subequations}
\begin{empheq}{align}
	\label{inn1}\tag{\textit{i}}
	\inn{x}{x} &\geq 0\\
	\label{inn2}\tag{\textit{ii}}
	\inn{x}{x} &= 0 \iff x = 0\\
	\label{inn3}\tag{\textit{iii}}
	\inn{x}{\varphi} &= \inn{\varphi}{x}^*\\
	\label{inn4}\tag{\textit{iv}}
	\inn{x}{a\varphi} &= a\inn{x}{\varphi}\\
	\label{inn5}\tag{\textit{v}}
	\inn{ax}{\varphi} &= a^*\inn{x}{\varphi}\\
	\label{inn6}\tag{\textit{vi}}
	\inn{ax}{\varphi} &= \inn{x}{a^*\varphi}\\
	\label{inn7}\tag{\textit{vii}}
	\inn{x_1 + x_2}{\varphi} &= \inn{x_1}{\varphi} + \inn{x_2}{\varphi}
\end{empheq}
\end{subequations}

\setcounter{equation}{1}
}

\item{
Let us write from now on in two dimensional vector space. Let $M$ be

\begin{equation}
\label{matrixM}
M = 
\begin{pmatrix}
m_1 & a\\
a^* & m_2
\end{pmatrix}
\text{, with } m_{i=1,2} \in \mathbb{R}
\end{equation}

\begin{enumerate}[label=\textit{\roman*.}]
\item{
Prove that $M = M^\dag$ ($M$ is hermitian).
}

\item{
Find the eigenvalues $\lambda_i$ of $Mx=\lambda x$, where $x$ is a 2-dimensional vector.
}

\item{
Prove that $\lambda_i$ are real.
}

\item{
Find the eigenvectors $x_i$ (corresponding to $\lambda_i$) such that are normalized to 1 $(x^\dag 
x = 1)$.
}

\item{
Prove that $x_i^*x_j\ (i\neq j)$ is zero, meaning the eigenvectors are orthonormal.
}

\end{enumerate}

\item{
Let $a$ be now real. Let $U$ be delivered as follows
\begin{equation}
\label{U}
U =
\begin{pmatrix}
\cos(\theta) & \sin(\theta)\\
-\sin(\theta) & \cos(\theta)
\end{pmatrix}
\end{equation}

\begin{enumerate}[label=\textit{\roman*.}]
\item{
Prove that $U$ is unitary $(U^\dag U = UU^\dag = 1)$.
}

\item{
Make a transformation ($M$ from 6.)
\begin{equation}
\label{transform}
\text{diag}(\lambda_1,\lambda_2) = UMU^\dag
\end{equation}
and find the angle $\theta$. What transformation on $x$ from $Mx=\lambda x$ is necessary to make it an eigenvector?
}
\end{enumerate}
}

\item{
Let $\hat{A}$ be an arbitrary (not necessarily hermitian) operator acting in the space of functions. Prove $\hat{O} = i(\hat{A} - \hat{A}^\dag)$ is hermitian $(\hat{O}=\hat{O}^\dag)$.
}

}


\end{enumerate}



\end{document}

%%%%%%%%% EXAMPLES %%%%%%%%%

%%%Example of boxed subequation%%%
%\begin{subequations}
%\begin{empheq}[box=\widefbox]{align}
%	\label{cos1}
%	&\cos(a) = \cos(b)\cos(c) + \sin(b)\sin(c)\cos(A) \\
%	\label{cos2}
%	&\cos(b) = \cos(a)\cos(c) \\
%	\label{cos3}
%	&\cos(c) = \cos(a)\cos(b) + \sin(a)\sin(b)\cos(C)
%\end{empheq}
%\end{subequations}