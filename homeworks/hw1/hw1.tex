\documentclass[11pt,letterpaper]{article}%letterpaper
% Choose language
\usepackage[english]{babel}

% Encoding options
\usepackage[T1]{fontenc}
\usepackage[utf8]{inputenc}
\usepackage{lmodern}

% Math and special stuff packages
\usepackage{amsmath}
\usepackage{mathtools}
\usepackage{amsfonts}
\usepackage{amssymb}
\usepackage{bigints}
\usepackage{graphicx}
% for listing with letters and changing number sizes
\usepackage{enumitem}
%\begin{enumerate}[label=\large{\textbf{\arabic*.}}]
%\begin{enumerate}[label=\large{\textbf{\alph*.}}]

% Selecting margins
\usepackage[left=1in,right=1in,top=1in,bottom=1in]{geometry}

% Griffiths options - change false for true
\iffalse
% For setting r as in Griffiths
%\usepackage{graphicx}
\def\rcurs{{\mbox{$\resizebox{.16in}{.08in}{\includegraphics{ScriptR}}$}}}
\def\brcurs{{\mbox{$\resizebox{.16in}{.08in}{\includegraphics{BoldR}}$}}}
\def\hrcurs{{\mbox{$\hat \rcurs$}}}
\def\hbrcurs{{\mbox{$\hat \brcurs$}}}
\fi
% Declare hat vectors
% ijk
\newcommand{\ihat}{\hat{\textbf{\i}}}
\newcommand{\jhat}{\hat{\textbf{\j}}}
\newcommand{\khat}{\hat{\textbf{k}}}
% xyzs
\newcommand{\xhat}{\hat{\textbf{x}}}
\newcommand{\yhat}{\hat{\textbf{y}}}
\newcommand{\zhat}{\hat{\textbf{z}}}
\newcommand{\shat}{\hat{\textbf{s}}}
% r theta phi
\newcommand{\rhat}{\hat{\textbf{r}}}
\newcommand{\thhat}{\hat{\boldsymbol{\theta}}}
\newcommand{\phihat}{\hat{\boldsymbol{\phi}}}


% Differentials
\usepackage{upgreek}
\newcommand{\dl}{\mathrm{x\textbf{l}}}
\newcommand{\da}{\mathrm{x\textbf{a}}}
\newcommand{\dv}{\mathrm{d}\uptau}
\newcommand{\dx}{\mathrm{d}x}
\newcommand{\dy}{\mathrm{d}y}
\newcommand{\dz}{\mathrm{d}z}


% Document information
\title{\textbf{Quantum Mechanics I HW1}}
\author{Diego Ramírez Milano (201214691)}
\date{Febrero 7 de 2016}

\begin{document}
\maketitle

Given the the wave function

\begin{equation}
\label{wavefunc}
\Psi(x,t) = e^{-iEt/\hbar}Ne^{-x^2/2\lambda^2}
\end{equation}

\begin{equation}
\label{wavefunc*}
\Psi^*(x,t) = e^{iEt/\hbar}Ne^{-x^2/2\lambda^2}
\end{equation}

Where $N$ is the normalization factor and $\lambda = \sqrt{\frac{\hbar}{m\omega}}$.\\The Hamiltonian operator for the harmonic oscillator reads

\begin{equation}
\label{hamil}
\hat{H} = -\frac{\hbar^2}{2m}\frac{\mathrm{d}^2}{\mathrm{d}x^2} + \frac{1}{2}kx^2 = -\frac{\hbar^2}{2m}\frac{\mathrm{d}^2}{\mathrm{d}x^2} + \frac{1}{2}m\omega^2x^2
\end{equation}

\textbf{Help}

\begin{equation}
\label{help1}
\bigintssss_0^{\infty}e^{-a^2x^2} = \frac{\sqrt{\pi}}{2a},\ a>0
\end{equation}

\begin{equation}
\label{help2}
\bigintssss_0^{\infty}x^2e^{-a^2x^2} = \frac{\sqrt{\pi}}{4a^3},\ a>0
\end{equation}

\begin{equation}
\label{prob}
\bigintssss_{\mathbb{R}}\Psi\Psi^*\dx = 1
\end{equation}

\begin{enumerate}
[label=\Large{\textbf{\arabic*.}}]
\item
{
Find the normalization factor $N$.

\begin{equation}
\label{p1:1}
\bigintssss_{\mathbb{R}}\Psi\Psi^*\dx = \bigintssss_{\mathbb{R}}e^{-iEt/\hbar}Ne^{-x^2/2\lambda^2}e^{iEt/\hbar}Ne^{-x^2/2\lambda^2}\dx = \bigintssss_{\mathbb{R}}N^2e^{-x^2/\lambda^2}\dx
\end{equation}

Now we use the substitution

\begin{equation}
\label{p1:2}
\begin{split}
u = \frac{x}{\lambda}\\
\lambda\mathrm{d}u = \dx
\end{split}
\end{equation}

\begin{equation}
\label{p1:3}
\begin{split}
\lambda N^2\bigintssss_{\mathbb{R}}e^{-u^2}\mathrm{d}u = 1\\
\lambda N^2\sqrt{\pi} = 1\\
\boxed{N = \frac{1}{\sqrt{\lambda \sqrt{\pi}}} = \sqrt[4]{\frac{m\omega}{\hbar\pi}}}
\end{split}
\end{equation}
}

\item
{
Calculate the probability for the particle to be in the interval $[0,\infty]$

\begin{equation}
\label{p2:1}
\bigintssss_0^{\infty}\Psi\Psi^*\dx = N^2\bigintssss_0^{\infty}e^{-x^2/\lambda^2}\dx
\end{equation}

Using same substitution as \eqref{p1:2} we get

\begin{equation}
\label{p2:2}
\begin{split}
\lambda N^2\bigintssss_0^{\infty}e^{-u^2}\dx = N^2\lambda\frac{\sqrt{\pi}}{2}\\
\boxed{P = \frac{\lambda}{\lambda\sqrt{\pi}}\frac{\sqrt{\pi}}{2} = \frac{1}{2}}
\end{split}
\end{equation}
}

\item
{
Calculate the values $\big<x\big>$, $\big<x^2\big>$, $\big<\hat{P}_x\big>$ and $\big<\hat{P}_x^2\big>$.

For $\big<x\big>$:
\begin{equation}
\label{p3:1}
\big<x\big> = \bigintssss_{\mathbb{R}}x\Psi\Psi^*\dx = N^2\bigintssss_{\mathbb{R}}xe^{-x^2/\lambda^2}\dx
\end{equation}

Now we use the substitution

\begin{equation}
\label{p3:2}
\begin{split}
u = \frac{x^2}{\lambda^2}\\
\mathrm{d}u = \frac{2x}{\lambda^2}\dx
\end{split}
\end{equation}

\begin{equation}
\label{p3:3}
\begin{split}
=& N^2\frac{\lambda^2}{2}\bigintssss_{\mathbb{R}}e^{-u}\mathrm{d}u = -N^2\frac{\lambda^2}{2}\left(e^{-u} \right)\\
=&-\frac{\lambda}{2\sqrt{\pi}}e^{-x^2/\lambda^2}\bigg|_{-\infty}^{\infty}\\
=& \boxed{-\frac{\lambda}{2\sqrt{\pi}}\left( 0 - 0\right) = 0}\\
\end{split}
\end{equation}

For $\big<x^2\big>$:
\begin{equation}
\label{p3:4}
\big<x^2\big> = \bigintssss_{\mathbb{R}}x^2\Psi\Psi^*\dx = N^2\bigintssss_{\mathbb{R}}x^2e^{-x^2/\lambda^2}\dx
\end{equation}

We take advantage now of help number 2 in equation \eqref{help2} to say:

\begin{equation}
\label{p3:5}
\begin{split}
N^2\bigintssss_{\mathbb{R}}x^2e^{-x^2/\lambda^2}\dx = N^2 \frac{\sqrt{\pi}\lambda^3}{2}\\
\boxed{\big<x^2\big> = \frac{\lambda^2}{2} = \frac{\hbar}{2m\omega}}
\end{split}
\end{equation}

For $\big<\hat{P}_x\big>$:

\begin{equation}
\label{p3:6}
\begin{split}
\big<\hat{P}_x\big> =& N^2\bigintssss_{\mathbb{R}}e^{-x^2/2\lambda^2}\left(-i\hbar \frac{\mathrm{d}}{\mathrm{d}x}e^{-x^2/2\lambda^2}\right)\dx\\
=& \frac{N^2i\hbar}{\lambda^2}\bigintssss_{\mathbb{R}}xe^{-x^2/\lambda^2}\dx\\
=& \boxed{\frac{N^2i\hbar}{\lambda^2}\big<x\big> = 0}
\end{split}
\end{equation}

For $\big<\hat{P}_x^2\big>$:

\begin{equation}
\label{p3:7}
\big<\hat{P}_x^2\big> = -\bigintssss_{\mathbb{R}}\hbar^2\frac{\mathrm{d}^2}{\mathrm{d}x^2}\Psi\Psi^*\dx = -N^2\hbar^2\bigintssss_{\mathbb{R}}e^{-x^2/2\lambda^2}\frac{\mathrm{d}^2}{\mathrm{d}x^2}e^{-x^2/2\lambda^2}\dx
\end{equation}

In a separate place we compute the derivatives

\begin{equation}
\label{deriv}
\begin{split}
\frac{\mathrm{d}}{\mathrm{d}x}e^{-x^2/2\lambda^2} =& -\frac{x}{\lambda}e^{-x^2/2\lambda^2}\\ 
\frac{\mathrm{d}^2}{\mathrm{d}x^2}e^{-x^2/2\lambda^2} =& \left( \frac{x^2 - \lambda^2}{\lambda^4} \right)e^{-x^2/2\lambda^2}
\end{split}
\end{equation}

Therefore using what we just computed we get:

\begin{equation}
\label{p3:8}
\begin{split}
\big<\hat{P}_x^2\big> =&  -N^2\hbar^2\bigintssss_{\mathbb{R}}e^{-x^2/2\lambda^2}\left( \frac{x^2 - \lambda^2}{\lambda^4} \right)e^{-x^2/2\lambda^2}\dx\\
=& -\frac{\hbar^2}{\lambda^4}\left( \bigintssss_{\mathbb{R}}N^2x^2e^{-x^2/2\lambda^2}\dx - \bigintssss_{\mathbb{R}}N^2\lambda^2e^{-x^2/2\lambda^2}\dx\right)\\
=& -\frac{\hbar^2}{\lambda^4}\left( \big<x^2\big> - \lambda^2  \right) = -\frac{\hbar^2}{\lambda^4}\left( \frac{\lambda^2}{2} - \lambda^2 \right)\\
=& \boxed{\frac{\hbar^2}{2\lambda^2} = \frac{mw}{2}}
\end{split}
\end{equation}
}

\item
{
Find the eigenvalue $E$ in $\hat{H}\Psi = E\Psi$

We now implement what is stated in equation \eqref{hamil} to the wave function, so we get

\begin{equation}
\label{p4:1}
\begin{split}
\hat{H}\Psi =& -\frac{\hbar^2}{2m}\frac{\mathrm{d}^2}{\mathrm{d}x^2}\Psi + \frac{1}{2}m\omega^2x^2\Psi\\
=& -\frac{\hbar^2}{2m}e^{-iEt/\hbar}Ne^{-x^2/2\lambda^2}\left( \frac{x^2 - \lambda^2}{\lambda^4} \right) + \frac{1}{2}m\omega^2x^2e^{-iEt/\hbar}Ne^{-x^2/2\lambda^2}\\
=& -\frac{\hbar^2}{2m}\Psi\left( \frac{x^2 - \lambda^2}{\lambda^4} \right) + \frac{1}{2}m\omega^2x^2\Psi\\
=& \Psi\left( -\frac{\hbar^2}{2m}\left( \frac{x^2 - \lambda^2}{\lambda^4} \right) + \frac{1}{2}m\omega^2x^2 \right)
\end{split}
\end{equation}

From here we can now solve for $E$:

\begin{equation}
\label{p4:2}
\begin{split}
E\Psi =& \Psi\left( -\frac{\hbar^2}{2m}\left( \frac{x^2 - \lambda^2}{\lambda^4} \right) + \frac{1}{2}m\omega^2x^2 \right)\\
E =& \boxed{-\frac{\hbar^2}{2m}\left( \frac{x^2 - \lambda^2}{\lambda^4} \right) + \frac{1}{2}m\omega^2x^2}
\end{split}
\end{equation}

Finally replacing $\lambda$ in equation \eqref{p4:2} we can simplify the expression to

\begin{equation}
\label{p4:3}
\begin{split}
E =& -\frac{\hbar^2m^2\omega^2x^2}{2m\hbar^2} + \frac{\hbar^2m\omega}{2m\hbar} + \frac{1}{2}m\omega^2x^2\\
=& -\frac{1}{2}m\omega^2x^2 + \frac{\hbar\omega}{2} + \frac{1}{2}m\omega^2x^2 = \boxed{\frac{\hbar\omega}{2}}
\end{split}
\end{equation}
}

\item
{
Calculate the probability current $j_x$

\begin{equation}
\label{p5:1}
\begin{split}
j_x =& \frac{i\hbar}{2m}\left(\Psi \frac{\mathrm{d}}{\dx}\Psi^* - \Psi^*\frac{\mathrm{d}}{\dx}\Psi \right)\\
=& \frac{i\hbar}{2m}\left(N^2e^{-x^2/2\lambda^2}\frac{\mathrm{d}}{\dx}e^{-x^2/2\lambda^2} - N^2e^{-x^2/2\lambda^2}\frac{\mathrm{d}}{\dx}e^{-x^2/2\lambda^2}\right)\\
=& \frac{i\hbar N^2}{2m}e^{-x^2/2\lambda^2}\left( \frac{\mathrm{d}}{\dx}e^{-x^2/2\lambda^2} - \frac{\mathrm{d}}{\dx}e^{-x^2/2\lambda^2} \right)\\
=&\ \boxed{0}
\end{split}
\end{equation}
}

\end{enumerate}

\end{document}