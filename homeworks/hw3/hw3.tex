\documentclass[11pt,letterpaper]{article}%letterpaper
% Choose language
\usepackage[english]{babel}

% Encoding options
\usepackage[T1]{fontenc}
\usepackage[utf8]{inputenc}
\usepackage{lmodern}

% Math and special stuff packages
\usepackage{amsmath}
\usepackage{mathtools}
\usepackage{amsfonts}
\usepackage{amssymb}
\usepackage{bigints}
\usepackage{graphicx}
\usepackage{siunitx}
\usepackage{empheq}
% for listing with letters and changing number sizes
\usepackage{enumitem}
%Use the following as arguments
%\begin{enumerate}[label=\large{\textbf{\arabic*.}}]
%\begin{enumerate}[label=\large{\textbf{\alph*.}}]

% Selecting margins
\usepackage[left=1in,right=1in,top=1in,bottom=1in]{geometry}


%%% NEW COMMANDS
%For boxing subequations in empheq environment 
\newcommand*\widefbox[1]{\fbox{\hspace{2em}#1\hspace{2em}}}

%Absolute Value
\newcommand\abs[1]{\left|#1\right|}

%%Inner product
\newcommand\inn[2]{\langle#1,#2\rangle}



%%%% CHANGING LABELS STYLE - may be used inside the document
%Sections labeled with roman numerals
\renewcommand\thesection{\Roman{section}}
%Equations labeled with roman numerals
%\renewcommand\theequation{\roman{equation}}
%Change equation counter
%\setcounter{equation}{4}

% Griffiths options - change false for true
\iffalse
% For setting r as in Griffiths
%\usepackage{graphicx}
\def\rcurs{{\mbox{$\resizebox{.16in}{.08in}{\includegraphics{ScriptR}}$}}}
\def\brcurs{{\mbox{$\resizebox{.16in}{.08in}{\includegraphics{BoldR}}$}}}
\def\hrcurs{{\mbox{$\hat \rcurs$}}}
\def\hbrcurs{{\mbox{$\hat \brcurs$}}}
\fi
% Declare hat vectors
% ijk
\newcommand{\ihat}{\hat{\textbf{\i}}}
\newcommand{\jhat}{\hat{\textbf{\j}}}
\newcommand{\khat}{\hat{\textbf{k}}}
% xyzs
\newcommand{\xhat}{\hat{\textbf{x}}}
\newcommand{\yhat}{\hat{\textbf{y}}}
\newcommand{\zhat}{\hat{\textbf{z}}}
\newcommand{\shat}{\hat{\textbf{s}}}
% r theta phi
\newcommand{\rhat}{\hat{\textbf{r}}}
\newcommand{\thhat}{\hat{\boldsymbol{\theta}}}
\newcommand{\phihat}{\hat{\boldsymbol{\phi}}}


% Differentials
\usepackage{upgreek}
\newcommand{\dl}{d\mathrm{\textbf{l}}}
\newcommand{\da}{d\mathrm{\textbf{a}}}
\newcommand{\dv}{d\uptau}
\newcommand{\dx}{\mathrm{d}x}
\newcommand{\dy}{\mathrm{d}y}
\newcommand{\dz}{\mathrm{d}z}


% Document information
\title{\textbf{Quantum Mechanics I - HW3}}
\author{Marek Nowakowski}
\date{\today}

\begin{document}
\maketitle

\begin{enumerate}[label=\Large{\textbf{\arabic*.}}]

\item{
A charged particle (charge $q$) us out in a constant electric field $E_0\xhat$. In addition it experiences a harmonic force in the same direction. Determine the energy eigenvalues and the eigenfunctions. To do so define a new variable
\begin{equation}
\label{variable}
\xi \equiv x - \frac{qE_0}{k}
\end{equation}
}

\item{
Find the eigenfunctions $\Psi(x,y)$ and the eigenvalues $E$ of a two dimensional harmonic oscillator with the Hamiltonian
\begin{equation}
\label{hamiltonian}
\hat{H} = \hat{H}_x + \hat{H_y = }\frac{\hat{P}_x^2}{2m} +  \frac{1}{2}m\omega x^2 + \frac{\hat{P}_y^2}{2m} +  \frac{1}{2}m\omega y^2
\end{equation}
Try a separation ansatz: $\Psi(x,y) = \Psi_x(x)\Psi_y(y)$.
}

\item{
Consider a particle with a charge $q>0$ moving under the three dimensional harmonic potential $V(r) = \dfrac{1}{2}m\omega r^2$ in an electric field $\vec{E} = E_0\xhat$. Find the eigenvalues by making a separation ansatz $\Psi(x,y,z) = \Psi_x(x)\Psi_y(y)\Psi_z(z)$.
}


\end{enumerate}



\end{document}

%%%%%%%%% EXAMPLES %%%%%%%%%

%%%Example of boxed subequation%%%
%\begin{subequations}
%\begin{empheq}[box=\widefbox]{align}
%	\label{cos1}
%	&\cos(a) = \cos(b)\cos(c) + \sin(b)\sin(c)\cos(A) \\
%	\label{cos2}
%	&\cos(b) = \cos(a)\cos(c) \\
%	\label{cos3}
%	&\cos(c) = \cos(a)\cos(b) + \sin(a)\sin(b)\cos(C)
%\end{empheq}
%\end{subequations}